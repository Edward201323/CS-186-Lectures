\section{Heap Files}

A \textbf{heap file} is the simplest file structure: records are stored with no particular ordering. As tuples are inserted or deleted, pages are allocated or freed as needed.

To support basic operations (scan all records, insert, delete, retrieve by record ID), we need to track which pages exist and how much free space each has.

\subsection{Design 1: Linked List}

\begin{figure}[H]
    \centering  \includegraphics[width=0.8\textwidth]{images/heap_linkedlist.png}
    \caption{Heap file organized as a doubly linked list}
    \label{fig:heap_linkedlist}
\end{figure}

\begin{itemize}
    \item A \textbf{header page} points to two doubly linked lists: one of \textbf{full pages} and one of \textbf{pages with free space}.
    \item Each page stores a pointer to the next page and the previous page.
\end{itemize}

\textbf{Problem:} To insert a record, we need a page with enough free space. We may have to walk many links in the free-space list before finding one that fits (e.g., for a 20-byte record, many free pages might not have 20 bytes left). Retrieval of a specific record by ID also requires walking through pages sequentially.

\subsection{Design 2: Page Directory}

\begin{itemize}
\begin{figure}[H]
    \centering  \includegraphics[width=0.4\textwidth]{images/heap_directory.png}
    \caption{Heap file organized using a directory}
    \label{fig:heap_map}
\end{figure}
    \item A \textbf{directory} stores, for each page, a pointer to the page and the amount of free space it has.
    \item To insert, look up the directory for a page with sufficient space and jump directly to it---no linked list traversal.
    \item The directory can itself span multiple pages (a ``directory of directories'') if needed, though too many layers hit diminishing returns.
\end{itemize}

\textbf{Improvement over linked list:} Insertion is faster because we can locate a suitable page in one lookup instead of walking a chain. Full pages still appear in the directory (with free space recorded as zero).